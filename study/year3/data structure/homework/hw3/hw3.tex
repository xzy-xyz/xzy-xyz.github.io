% Modified based on Xiaoming Sun's template
\documentclass{article}
\usepackage{amsmath,amsfonts,amsthm,amssymb}
\usepackage[slantfont,boldfont]{xeCJK}
\usepackage{fontspec}
\usepackage{setspace}
\usepackage{fancyhdr}
\usepackage{lastpage}
\usepackage{extramarks}
\usepackage{algorithm}
\usepackage{chngpage}
\usepackage{algpseudocode}
\usepackage{caption}
\usepackage{lipsum}
\usepackage{soul,color}
\usepackage{listings}
\usepackage[dvipsnames]{xcolor}
\newtheorem{theorem}{Theorem}
\newtheorem{lemma}{Lemma}
\newtheorem{definition}{Definition}
\newtheorem{corollary}{Corollary}
\DeclareCaptionFormat{algor}{%
  \hrulefill\par\offinterlineskip\vskip1pt%
    \textbf{#1#2}#3\offinterlineskip\hrulefill}
\DeclareCaptionStyle{algori}{singlelinecheck=off,format=algor,labelsep=space}
\captionsetup[algorithm]{style=algori}
\definecolor{mygreen}{rgb}{0,0.6,0}  
\definecolor{mygray}{rgb}{0.5,0.5,0.5}  
\definecolor{mymauve}{rgb}{0.58,0,0.82} 
\definecolor{keywordcolor}{RGB}{0,0,255} % 设置关键字颜色为蓝色
\usepackage{graphicx,float,wrapfig}
\newcommand{\Class}{数据结构}

% Homework Specific Information. Change it to your own
\newcommand{\Title}{Homework 3}

\bibliographystyle{plain}% the recommended bibstyle
% In case you need to adjust margins:
\topmargin=-0.45in      %
\evensidemargin=0in     %
\oddsidemargin=0in      %
\textwidth=6.5in        %
\textheight=9.0in       %
\headsep=0.25in         %

% Setup the header and footer
\pagestyle{fancy}                                                       %
\chead{\Title}  %
\rhead{\firstxmark}                                                     %
\lfoot{\lastxmark}                                                      %
\cfoot{}                                                                %
\rfoot{Page\ \thepage\ of\ \protect\pageref{LastPage}}                          %
\renewcommand\headrulewidth{0.4pt}                                                                       %

%%%%%%%%%%%%%%%%%%%%%%%%%%%%%%%%%%%%%%%%%%%%%%%%%%%%%%%%%%%%%
% Some tools
\newcommand{\enterProblemHeader}[1]{\nobreak\extramarks{#1}{}\nobreak%
                                    \nobreak\extramarks{#1}\nobreak}%
\newcommand{\exitProblemHeader}[1]{\nobreak\extramarks{#1}\nobreak%
                                   \nobreak\extramarks{#1}{}\nobreak}%

\newcommand{\homeworkProblemName}{}%
\newcounter{homeworkProblemCounter}%
\newenvironment{homeworkProblem}[1][Problem \arabic{homeworkProblemCounter}]%
  {\stepcounter{homeworkProblemCounter}%
   \renewcommand{\homeworkProblemName}{#1}%
   \section*{\homeworkProblemName}%
   \enterProblemHeader{\homeworkProblemName}}%
  {\exitProblemHeader{\homeworkProblemName}}%

\newcommand{\homeworkSectionName}{}%
\newlength{\homeworkSectionLabelLength}{}%
\newenvironment{homeworkSection}[1]%
  {% We put this space here to make sure we're not connected to the above.

   \renewcommand{\homeworkSectionName}{#1}%
   \settowidth{\homeworkSectionLabelLength}{\homeworkSectionName}%
   \addtolength{\homeworkSectionLabelLength}{0.25in}%
   \changetext{}{-\homeworkSectionLabelLength}{}{}{}%
   \subsection*{\homeworkSectionName}%
   \enterProblemHeader{\homeworkProblemName\ [\homeworkSectionName]}}%
  {\enterProblemHeader{\homeworkProblemName}%

   % We put the blank space above in order to make sure this margin
   % change doesn't happen too soon.
   \changetext{}{+\homeworkSectionLabelLength}{}{}{}}%

\newcommand{\Answer}{\ \\\textbf{Answer:} }
\newcommand{\Acknowledgement}[1]{\ \\{\bf Acknowledgement:} #1}

%%%%%%%%%%%%%%%%%%%%%%%%%%%%%%%%%%%%%%%%%%%%%%%%%%%%%%%%%%%%%


%%%%%%%%%%%%%%%%%%%%%%%%%%%%%%%%%%%%%%%%%%%%%%%%%%%%%%%%%%%%%
% Make title
\title{\textmd{\bf \Class: \Title}}
\date{\today}
\author{张霄~~~~2019030045}
%%%%%%%%%%%%%%%%%%%%%%%%%%%%%%%%%%%%%%%%%%%%%%%%%%%%%%%%%%%%%
\definecolor{dkgreen}{rgb}{0,0.6,0}
\definecolor{gray}{rgb}{0.5,0.5,0.5}
\definecolor{mauve}{rgb}{0.58,0,0.82}

\lstset{frame=none,
  language = C++,
  %backgroundcolor=\color[RGB]{245,245,244},
  %rulesepcolor= \color{ red!20!green!20!blue!20} ,
  aboveskip=3mm,
  belowskip=3mm,
  showstringspaces=false,
  columns=flexible,
  basicstyle={\small\ttfamily},
  numbers=left,
  numberstyle=\tiny\color{gray},
  commentstyle=\itshape\color{black!50!white}, % 设置注释样式,斜体,浅灰色
  stringstyle=\color{mauve},
  breaklines=true,
  breakatwhitespace=true,
  tabsize=3,
  keywordstyle=[1]\bfseries\color{NavyBlue},
  keywordstyle=[2]\bfseries\color{PineGreen!90!black},
  keywordstyle=[3]\bfseries\color{Rhodamine}
}
\lstdefinestyle{customcpp}{
    morekeywords=[2]{succ, pred},
    morekeywords=[3]{NULL}
}
\begin{document}
\begin{spacing}{1.1}
\maketitle \thispagestyle{empty}

%%%%%%%%%%%%%%%%%%%%%%%%%%%%%%%%%%%%%%%%%%%%%%%%%%%%%%%%%%%%%
% Begin edit from here


\begin{homeworkProblem}[Problem 1]
1).
\begin{lstlisting}[style = customcpp]
x->succ = x->succ->succ; // x的后继指向x后继的后继,即跳过x后面的一位
\end{lstlisting}
2).
\begin{lstlisting}[style = customcpp]
t->succ = x->succ;  // 将新结点t的后继指针指向x结点的后继结点,即将t插入到x结点后面
x->succ = t;  // 将x结点的后继指针指向新结点t,完成插入操作
\end{lstlisting}
3).
\begin{lstlisting}[style = customcpp]
ListNode* fun(ListNode* x){  // x指向列表的头结点,函数返回反转后的列表的头结点
    ListNode* a = x;  // 定义指针a指向原始列表的头结点
    ListNode* b = NULL;  // 定义指针b为NULL,用于存储反转后列表的头结点
    while (a != NULL){  // 当原始列表未遍历完时
        ListNode* c = a->succ;  // 将指针c指向a的后继结点,保存起来以便后续操作
        a->succ = b;  // 将a的后继指针指向b,完成反转操作
        b = a;  // 更新b指针,使其指向当前反转后的列表的头结点
        a = c;  // 将a指向原始列表中的下一个结点,准备进行下一次反转操作
    }
    return b;  // 返回反转后的列表的头结点
}

\end{lstlisting}
\end{homeworkProblem}
\begin{homeworkProblem}[Problem 2]
1). 使用两个指针,他们之间的间距为k,当快指针到达列表的末尾时,慢指针将指向倒数第k个节点。这里我们最多遍历一次列表,时间复杂度为 $\mathcal{O}(n)$,使用了常量级的额外指针,空间复杂度为 $\mathcal{O}(1)$。
\begin{lstlisting}[style = customcpp]
// 定义单向列表的节点结构
struct ListNode {
    int val;
    ListNode* succ;
    ListNode(int x) : val(x), succ(NULL) {}
};
ListNode* findKthFromEnd(ListNode* head, int k) {
    // 定义快指针和慢指针,初始时都指向列表的头节点
    ListNode* fast = head;
    ListNode* slow = head;
    if (k == 0){
    // 单独列出来,因为如果需要输出最后一个节点用双指针相当于遍历了两次
        while(fast->succ != NULL){
          fast = fast->succ;
        }
        return fast;
    }
    // 将快指针移动k步,使其与慢指针相隔k个节点
    for (int i = 0; i < k; ++i) {
        // 如果快指针已经到达列表末尾,说明列表长度小于n,无法找到倒数第n个节点,返回NULL
        if (fast == NULL) {
            return NULL;
        }
        fast = fast->succ;
    } 
    // 同时移动快慢指针,直到快指针到达列表末尾
    while (fast->succ != NULL) {
        fast = fast->succ;
        slow = slow->succ;
    }
    // 此时慢指针指向倒数第k个节点,返回慢指针指向的节点即可
    return slow;
}
\end{lstlisting}
~~\\
2). 使用两个指针,一个快指针和一个慢指针,它们之间的间隔为k+1。当快指针到达列表的末尾时,慢指针将指向要删除节点的前一个节点。然后通过修改慢指针的succ指针来删除倒数第k个节点。这里我们最多遍历一次列表,因此时间复杂度为 $\mathcal{O}(n)$,使用了常量级的额外指针,空间复杂度为 $\mathcal{O}(1)$。
\begin{lstlisting}[style = customcpp]
// 定义单向列表的节点结构
struct ListNode {
    int val;
    ListNode* succ;
    ListNode(int x) : val(x), succ(NULL) {}
};
ListNode* removeFromEnd(ListNode* head, int k) {
    // 定义快指针和慢指针,初始时都指向列表的头节点
    ListNode* fast = head;
    ListNode* slow = head;
    // 将快指针移动k+1步,使其与慢指针相隔k个节点
    for (int i = 0; i <= k; ++i) {
        // 如果快指针已经到达列表末尾,说明列表长度小于k,直接返回头节点的下一个节点
        if (fast->succ == NULL) {
            return head->succ;
        }
        fast = fast->succ;
    }
    // 同时移动快慢指针,直到快指针到达列表末尾
    while (fast->succ != NULL) {
        fast = fast->succ;
        slow = slow->succ;
    }
    // 此时慢指针指向倒数第k+1个节点,修改其succ指针,删除倒数第k个节点
    slow->succ = slow->succ->succ;
    return head;// 返回列表的头节点
}
\end{lstlisting}
\end{homeworkProblem}

\begin{homeworkProblem}[Problem 3]
仍旧使用双指针的方法,两个列表的头部分别为 $\bf len1,~len2$,然后遍历计算两个列表的长度。因为它们在第一个公共节点之后的部分应该一样,所以我们计算两个列表的长度差,把较长列表的指针 $\bf node1$ 先向后移 $\bf abs(len1-len2)$,较短的列表指针 $\bf node2$ 仍旧指在开头。接着同时移动两个指针,他们的 $\bf succ$ 指向同一个节点为止,输出节点即输出指向这个节点的指针。由于我们最多需要遍历两个列表各两次,因此时间复杂度是 $\mathcal{O}(n)$,而因为只用了常量级的额外空间,空间复杂度为 $\mathcal{O}(1)$。
\begin{center}
\captionof{algorithm}{problem 3}
  \begin{algorithmic}
    \State $len1~=~0$
    \State $len2~=~0$
    \State $node1~=~head1$
    \State $node2~=~head2$
    \While{$node1~is~not~$NULL}:\Comment 计算列表1的长度
        \State $len1++$
        \State $node1 ~=~ node1\rightarrow succ$
    \EndWhile
    \While{$node2~is~not~$NULL}:\Comment 计算列表2的长度
        \State $len2++$
        \State $node2 ~=~ node2\rightarrow succ$
    \EndWhile
    \Comment 调整列表起始位置
    \State $node1 ~=~ head1$
    \State $node2 ~=~ head2$
    \State $diff ~=~ abs(len1 ~- ~len2)$
    \If {$len1 ~>~ len2$}:
        \For {$i~from~1~to~diff$}:
            \State $node1 ~=~ node1\rightarrow succ$
        \EndFor
    \Else:
        \For{$i~from~1~to~diff$}:
            \State $node2 ~=~ node2\rightarrow succ$
        \EndFor
    \EndIf
    \State \Comment 同步移动两个指针,直到找到第一个公共节点
    \While {$node1~is~not~NULL~and~node2~is~not~$NULL}:
        \If{$node1 ~==~ node2$}:
          \State \textbf{return} $node1$ \Comment 找到第一个公共节点
        \EndIf
        \State $node1 ~=~ node1\rightarrow succ$
        \State $node2 ~=~ node2\to succ$
    \EndWhile
    \State \textbf{return} NULL \Comment 没有找到公共节点
    \State
\hrule
  \end{algorithmic}
\end{center}
\end{homeworkProblem}

\begin{homeworkProblem}[Problem 4]
\begin{center}
\captionof{algorithm}{problem 4}
\begin{algorithmic}
\Function {BinarySearch}{vector, target}:
    \State $left ~=~ 0$
    \State $right ~=~ length(vector) ~-~ 1$
    
    \While{$left~ \leq~ right$}:
        \State $mid ~=~ left ~+ ~(right ~- ~left) ~/~ 2$
        
        \If{$vector[mid] ~==~ target$}:
            \State \textbf{return} $mid$  \Comment 找到目标,返回目标元素的索引
        \ElsIf{$vector[mid] ~<~ target$}:
            \State $left ~=~ mid ~+~ 1$  \Comment 目标在右半部分,缩小搜索范围
        \Else:
            \State $right ~= ~mid ~-~ 1$  \Comment 目标在左半部分,缩小搜索范围
        \EndIf
    \EndWhile
    \State \textbf{return} -1  \Comment 没有找到目标,返回-1
\EndFunction
\State
\hrule
\end{algorithmic}
\end{center}
因为是一个已经排列好的向量,每次比较都将搜索范围缩小为原来的一半。在最坏情况下,二分查找最多需要$\mathcal{O}(\log n)$次比较来找到目标元素
\end{homeworkProblem}

\begin{homeworkProblem}
\begin{center}
\captionof{algorithm}{problem 5}
\begin{algorithmic}
\Function{BinarySearch}{head, target}
    \State $left ~=~$ head
    \State $right ~=~$ NULL  \Comment 右指针初始化为NULL,表示列表结束
    
    \While{$left \neq$ right and $left \neq$ NULL}
        \State $mid ~=~$ FindMiddle($left$, $right$)  \Comment 找到中间节点
        \If{$mid\rightarrow\text{val} = \text{target}$}
            \State \textbf{return} $mid$  \Comment 找到目标节点,返回节点指针
        \ElsIf{$mid\rightarrow\text{val} < \text{target}$}
            \State $left ~=~ mid\rightarrow\text{next}$  \Comment 目标在右半部分,移动左指针
        \Else
            \State $right ~=~ mid$  \Comment 目标在左半部分,调整右指针
        \EndIf
    \EndWhile
    \State \textbf{return} NULL  \Comment 没有找到目标节点,返回NULL
\EndFunction

\Function{FindMiddle}{left, right}
    \State $slow ~=~$ left
    \State $fast ~=~$ left
    
    \While{$fast \neq$ right and $fast\rightarrow\text{next} \neq$ right}
        \State $slow ~=~ slow\rightarrow\text{next}$
        \State $fast ~=~ fast\rightarrow\text{next}\rightarrow\text{next}$
    \EndWhile
    
    \State \textbf{return} $slow$  \Comment 返回中间节点指针
\EndFunction
\State 
\hrule
\end{algorithmic}
\end{center}
由于每一次比较都将搜索范围缩小为原来的一半,所以在有序单向列表上进行二分查找的时间复杂度是$\mathcal{O}(\log n)$,其中n是列表的长度。但找到单向列表的中值需要对列表进行遍历,所以总的时间复杂度是 $\mathcal{O}(n)$。
\end{homeworkProblem}
% End edit to here
%%%%%%%%%%%%%%%%%%%%%%%%%%%%%%%%%%%%%%%%%%%%%%%%%%%%%%%%%%%%%
\end{spacing}
\end{document}

%%%%%%%%%%%%%%%%%%%%%%%%%%%%%%%%%%%%%%%%%%%%%%%%%%%%%%%%%%%%%
