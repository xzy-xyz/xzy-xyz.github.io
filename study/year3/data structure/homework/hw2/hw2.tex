% Modified based on Xiaoming Sun's template
\documentclass{article}
\usepackage{amsmath,amsfonts,amsthm,amssymb}
\usepackage[slantfont,boldfont]{xeCJK}
\usepackage{fontspec}
\usepackage{setspace}
\usepackage{fancyhdr}
\usepackage{lastpage}
\usepackage{extramarks}
\usepackage{algorithm}
\usepackage{chngpage}
\usepackage{algpseudocode}
\usepackage{caption}
\usepackage{lipsum}
\usepackage{soul,color}
\usepackage{listings}
\usepackage{xcolor}
\newtheorem{theorem}{Theorem}
\newtheorem{lemma}{Lemma}
\newtheorem{definition}{Definition}
\newtheorem{corollary}{Corollary}
\DeclareCaptionFormat{algor}{%
  \hrulefill\par\offinterlineskip\vskip1pt%
    \textbf{#1#2}#3\offinterlineskip\hrulefill}
\DeclareCaptionStyle{algori}{singlelinecheck=off,format=algor,labelsep=space}
\captionsetup[algorithm]{style=algori}
\definecolor{mygreen}{rgb}{0,0.6,0}  
\definecolor{mygray}{rgb}{0.5,0.5,0.5}  
\definecolor{mymauve}{rgb}{0.58,0,0.82} 


\usepackage{graphicx,float,wrapfig}
\newcommand{\Class}{数据结构}

% Homework Specific Information. Change it to your own
\newcommand{\Title}{Homework 2}

\bibliographystyle{plain}% the recommended bibstyle
% In case you need to adjust margins:
\topmargin=-0.45in      %
\evensidemargin=0in     %
\oddsidemargin=0in      %
\textwidth=6.5in        %
\textheight=9.0in       %
\headsep=0.25in         %

% Setup the header and footer
\pagestyle{fancy}                                                       %
\chead{\Title}  %
\rhead{\firstxmark}                                                     %
\lfoot{\lastxmark}                                                      %
\cfoot{}                                                                %
\rfoot{Page\ \thepage\ of\ \protect\pageref{LastPage}}                          %
\renewcommand\headrulewidth{0.4pt}                                      %
\renewcommand\footrulewidth{0.4pt}                                      %

%%%%%%%%%%%%%%%%%%%%%%%%%%%%%%%%%%%%%%%%%%%%%%%%%%%%%%%%%%%%%
% Some tools
\newcommand{\enterProblemHeader}[1]{\nobreak\extramarks{#1}{#1 continued on next page\ldots}\nobreak%
                                    \nobreak\extramarks{#1 (continued)}{#1 continued on next page\ldots}\nobreak}%
\newcommand{\exitProblemHeader}[1]{\nobreak\extramarks{#1 (continued)}{#1 continued on next page\ldots}\nobreak%
                                   \nobreak\extramarks{#1}{}\nobreak}%

\newcommand{\homeworkProblemName}{}%
\newcounter{homeworkProblemCounter}%
\newenvironment{homeworkProblem}[1][Problem \arabic{homeworkProblemCounter}]%
  {\stepcounter{homeworkProblemCounter}%
   \renewcommand{\homeworkProblemName}{#1}%
   \section*{\homeworkProblemName}%
   \enterProblemHeader{\homeworkProblemName}}%
  {\exitProblemHeader{\homeworkProblemName}}%

\newcommand{\homeworkSectionName}{}%
\newlength{\homeworkSectionLabelLength}{}%
\newenvironment{homeworkSection}[1]%
  {% We put this space here to make sure we're not connected to the above.

   \renewcommand{\homeworkSectionName}{#1}%
   \settowidth{\homeworkSectionLabelLength}{\homeworkSectionName}%
   \addtolength{\homeworkSectionLabelLength}{0.25in}%
   \changetext{}{-\homeworkSectionLabelLength}{}{}{}%
   \subsection*{\homeworkSectionName}%
   \enterProblemHeader{\homeworkProblemName\ [\homeworkSectionName]}}%
  {\enterProblemHeader{\homeworkProblemName}%

   % We put the blank space above in order to make sure this margin
   % change doesn't happen too soon.
   \changetext{}{+\homeworkSectionLabelLength}{}{}{}}%

\newcommand{\Answer}{\ \\\textbf{Answer:} }
\newcommand{\Acknowledgement}[1]{\ \\{\bf Acknowledgement:} #1}

%%%%%%%%%%%%%%%%%%%%%%%%%%%%%%%%%%%%%%%%%%%%%%%%%%%%%%%%%%%%%


%%%%%%%%%%%%%%%%%%%%%%%%%%%%%%%%%%%%%%%%%%%%%%%%%%%%%%%%%%%%%
% Make title
\title{\textmd{\bf \Class: \Title}}
\date{\today}
\author{张霄~~~~2019030045}
%%%%%%%%%%%%%%%%%%%%%%%%%%%%%%%%%%%%%%%%%%%%%%%%%%%%%%%%%%%%%
\definecolor{dkgreen}{rgb}{0,0.6,0}
\definecolor{gray}{rgb}{0.5,0.5,0.5}
\definecolor{mauve}{rgb}{0.58,0,0.82}

\lstset{frame=tb,
  language=Python,
  aboveskip=3mm,
  belowskip=3mm,
  showstringspaces=false,
  columns=flexible,
  basicstyle={\small\ttfamily},
  numbers=none,
  numberstyle=\tiny\color{gray},
  keywordstyle=\color{blue},
  commentstyle=\color{dkgreen},
  stringstyle=\color{mauve},
  breaklines=true,
  breakatwhitespace=true,
  tabsize=3
}
\begin{document}
\begin{spacing}{1.1}
\maketitle \thispagestyle{empty}

%%%%%%%%%%%%%%%%%%%%%%%%%%%%%%%%%%%%%%%%%%%%%%%%%%%%%%%%%%%%%
% Begin edit from here


\begin{homeworkProblem}[Problem 1]
当 $n<2^k$时,作用于初始为0的计数器 $n$ 次,对于 $i>\lfloor \log n\rfloor$ 的位来说始终没有影响,但是对 $i \leq \lfloor \log n \rfloor$ 的位置来说,$A[0]$ 会变化 $n$ 次,$A[1]$ 会变化 $\lfloor n / 2\rfloor$ 次,以此类推,$A[i]$ 会变化 $\lfloor n/2^i\rfloor$ 次,因此总的代价满足:
$$\sum_{i=0}^{\lfloor \log n \rfloor} \lfloor\frac{n}{2^i}\rfloor < n\sum_{i=0}^{\infty} \frac{1}{2^i}=2n$$
所以从0开始调用 $n$ 次 INCREMENT 操作的代价是 $\mathcal{O}(n)$ 的,那么平均代价就是:
$$\mathcal{O}(n)/n = \mathcal{O}(1)$$
当 $n\geq 2^k$时,所有$k$都会被占满,$A[0]$ 会变化 $2^k$ 次,$A[1]$ 会变化 $2^{k-1}$ 次,以此类推,$A[k]$ 会变化 $1$ 次,因此总的代价满足:
$$\sum_{i=0}^{k} 2^i =2^{k+1}-1\leq 2n-1=\mathcal{O}(n)$$
平均代价为:
$$\mathcal{O}(n)/n = \mathcal{O}(1)$$
\end{homeworkProblem}

\begin{homeworkProblem}[Problem 4]
 先对这 $n$ 个长度进行排序,这个过程需要 $\mathcal{O}(n\log n)$复杂度。之后想要计算个数仍旧需要遍历,可以使用二分查找的方法。假设我们排序的结果是从小到大,存在数组 $A$中(下标从$0$到$n-1$),那么用一个$for$循环,从下标为 $n-1$的木棒开始到下标为$2$的木棒截止。假设我们现在循环到了下标为$i$的木棒,$left$指向index为$0$的位置,$right$指向index为$i-1$的位置。接着判断条件$A[left]+A[right]>A[i]$,如果满足条件那么就给我们的结果加上 $right-left$,同时让 $right$向左移一位;如果不满足条件我们就让$left$向右移一位,直到$left\geq right$为止。 
\begin{center}
  \captionof{algorithm}{Binary Search}
  \begin{algorithmic}[1]
    \State A[n] \Comment{Put n numbers in an array A}
    \State $Arrays.sort(A)$\Comment{Sort array A from small to large}
    \State $nums=0$\Comment{The number of triangles that can be formed}
    \For{$i~from~n-1~to~2$}
      \State $left~=~0,right~=~i~-~1$
      \While{$left~<~right$}
        \If{$A[left]~+~A[right]~>~A[i]$}
          \State $nums~+=~right~-~left$
          \State $right~=~right~-~1$
        \Else
          \State $left~=~left~+~1$
        \EndIf
      \EndWhile
    \EndFor
    \State \textbf{return} $nums$
  \end{algorithmic}
\end{center}
 
 最终得到的$nums$就是总的三角形数,该部分复杂度为$\mathcal{O}(n^2)$。总的复杂度为
 $$\mathcal{O}(n\log n)+\mathcal{O}(n^2)=\mathcal{O}(n^2)$$
\end{homeworkProblem}

\begin{homeworkProblem}[Problem 5]
\paragraph{容量递增策略}$\mathrm{T^*~oldElem = \_elem; \_elem=new~T[\_capacity~+=~INCREMENT]}$\\
最坏情况下在初始容量为0的空向量中连续插入 $n=m*I>>2$个元素,而无删除操作,因此在第 $1,I+1,2I+1,\cdots$次插入时需要扩容。每次扩容复制原向量的时间成本依次为 $\mathcal{O}(0),\mathcal{O}(I),\mathcal{O}(2I),\cdots,\mathcal{O}((m-1)I)$\\
累计扩容时间为
$$\mathcal{O}(0)+\mathcal{O}(I)+\cdots+\mathcal{O}((m-1)I)=\mathcal{O}(n^2)$$
分摊扩容时间为
$$\mathcal{O}(n^2)/n=\mathcal{O}(n)$$
空间利用率考虑向量实际规模与其内部数组容量的比值,即装填因子),显然当 $n$ 足够大的时候,装填因子趋近于 $100\%$。
\paragraph{容量倍增策略}$\mathrm{T^*~oldElem = \_elem;\_elem=new~T[\_capacity <<= 1];}$\\
最坏情况在初始容量为1的满向量中,连续插入 $n=2^m>>2$个元素而无删除操作,在第$1,2,4,8,16,\cdots$次插入时需要扩容。每次扩容的时间成本为 $1,2,4,8,16,\cdots,2^{m-1},2^{m}=n$\\
累计扩容时间为
$$\mathcal{O}(1)+\mathcal{O}(2)+\mathcal{O}(4)+\cdots+\mathcal{O}(2^{m-1})+\mathcal{O}(2^m)=\mathcal{O}(2^{m+1}-1)=\mathcal{O}(n)$$
分摊扩容时间为
$$\mathcal{O}(n)/n=\mathcal{O}(1)$$
空间利用率大于 $50\%$\\ \\
综上可以看出,容量倍增的时间成本较容量递增更小,但空间利用率更大,可以认为容量倍增是是在空间上适当牺牲换取时间上的巨大收益。
% paragraph paragraph_name (end)
\end{homeworkProblem}

% End edit to here
%%%%%%%%%%%%%%%%%%%%%%%%%%%%%%%%%%%%%%%%%%%%%%%%%%%%%%%%%%%%%
\end{spacing}
\end{document}

%%%%%%%%%%%%%%%%%%%%%%%%%%%%%%%%%%%%%%%%%%%%%%%%%%%%%%%%%%%%%
